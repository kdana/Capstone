\documentclass[11pt]{article}
\bibliographystyle{plain}
\usepackage{graphicx}
\usepackage[margin=.75in]{geometry}
\usepackage{amsmath}
\usepackage{comment}
\usepackage{setspace}
\usepackage{indentfirst}
\long\def\/*#1*/{}
\doublespace

\begin{document}
\title{\#FriendFinder}
\author{Tyler Allen, Karen Dana}
\maketitle

\section{Overview}

Our goal is to create a social networking application for the Android mobile 
device platform tentatively called `Friend Finder'. 
 The users of this application will have profile information including their 
name, schedule, `likes', `dislikes', and other useful information; parts of 
this data will be hidden from other users, but will still be used in determining if a 
user should receive certain notifications or if a user should be flagged 
`busy' or `available'.
The application will also use GPS location services to provide the 
location of events and students who wish to make their location public. This 
will provide students with a fun and easy-to-use service that will eliminate 
the hassle of individually inviting people to events, finding event locations, 
as well as locating a friend for coffee.\\

The primary deliverable of this project will be the Android application; we 
will be using the Android SDK and fundamental aspects of Android development 
pulled from the Android Development course in order to create this application. 
For this application to function, we will need users to be able to login to 
their profiles; this will require us to have an SQL-based back-end database 
that contains user credentials and profile information, as well as a database 
connector API such as JDBC. We will be using an external server to house and 
maintain the database.  This project will require a full spectrum of 
software engineering and development tools and techniques. We will need to 
explore a variety of both Android technologies, such as GPS and Android 4.0 
technologies, as well as software libraries for database connections and 
password encryption; it may also be necessary to explore algorithms for 
handling large data requests via the internet.\\

	In addition, we are to write a Campus Emergency Application targeted at the
    Android Platform for Western Carolina University. This application will 
    provide relevant Emergency Information for students, such as contact 
    information for 911, suicide hotlines, sexual assault hotlines, 
    campus police, and campus medical facilities. This application is to be
    Western Carolina University themed with the campus police as our 
    clients.

\section{Problem Statement}

The purpose of this project is to develop a social networking application 
targeted at the Android Mobile platform which will provide context- and 
location-based communication and event scheduling with other users backed by 
an external database, maintaining user profiles and 
other information, to simplify data processing. We are also to design a 
Campus Safety application targeted at the Android platform in order to provide
public safety information, from both on-campus and off-campus sources, 
to students in an easily accessible manner.


\section{Requirements Specification}

\subsection{Campus Police Application}
\begin{itemize}
\item This application must provide phone numbers, web pages, and other pertinent 
contact information for on-campus and local emergency resources in a 
well-organized, functional manner.
 \item This application may provide the ability to call/contact these resources 
directly from the app if it can be done in such a way that is not intrusive 
and is not likely to happen by accident, but is still convenient enough to be 
usable.
\end{itemize}


\subsection{Friend Finder Application}
\begin{itemize}
\item The android application should provide the ability to create a persistent 
profile containing information, of which some will be publicly available and some 
will be private. We will investigate using facebook authentication 
for this app, as it seems to be both popular and reduce the complexity of 
building a login system \cite{capstone_facebook}.

\item The android application should allow a user to join groups, add `likes' to 
their profile, and broadcast/receive events based on these likes and groups. 
Users should be able to create likes/groups if a reasonable protocol for doing
so becomes apparent while maintaining security and without overcomplicating
the app.

\item The android application should provide the ability for users to search/query 
for other users, `likes', and groups available within the app. The user should 
    also be able to upload their schedule; this will allow the application to 
    deactivate itself or take other action when the user is `busy' according to
    their schedule.

\item The user should be able to have a `friends' list, a special group which allows 
viewing of additional profile information such as the user's class/work 
schedule. This will also allow the user to broadcast their location to their 
friends, making meetups easier. GPS is notoriously known for using large amounts 
of power, draining the battery in a short amount of time. We will be investigating
methods for efficiently making use of the GPS transmitter as well as using 
other technologies in place of GPS in order to conserve power \cite{conserve_gps} \cite{gps_alternatives}.

\item The application should provide different levels of events and determine if 
events warrant a push notification versus a log under their respective page.
The server should provide a stable, consistent database of user credentials, 
profiles, and other information.

\item The server should be able to provide information about a user without revealing 
personal information (i.e. extract `busy' status from a user's schedule and 
deliver it to other users, 
without revealing the actual details of the schedule).

\item The server should provide user authentication; this user authentication will 
be encrypted if it is feasible to include this within the scope of this project.
\end{itemize}






\section{Testing Plan}

    Testing will be completed as the application is developed. We will create 
    Regression/Integration tests using the Android TestSuite - a derivative of 
    JUnit. An adequate server-side unit testing framework will be selected 
    based on the final language decision; presently, there is no concrete 
    language decision for the server while we weigh our concerns about 
    performance, usability, and database and API availability.  Testing will 
    also include an evaluation of the user experience on Android devices 
    of multiple sizes, and conceivably performance concerns. Ideally, we will be
    able to generate and/or simulate a large number of registered users as well
    as simulated users to test usability. Our server will be the server Polaris. 
    Our test android devices will be a Samsung Galaxy Neo, an LG G2, 
    a Google Nexus 7 (2012), and any other devices we can allocate during 
    the course of this project. We will be targeting Android Version 2.3 for 
    the Campus Police application and Android Version 4.0 for the `Friend 
    Finder' application.

\section{Schedule of Completion}

\textbf{Tuesday, September 16, 2014}
\\
\indent We will have completed the emergency application for the Western 
Carolina campus police. The planning stage for the second part of our capstone
will also be finished; this includes the layouts for the Android application
and UML diagrams for the entire `Friend Finder' project.\\


\textbf{Tuesday, September 30, 2014}
\\  \indent  At this meeting, we will have the skeleton of the Android application 
    finished. This contains the screens that will be used, including the 
    graphics, layouts, styling, and non event-driven buttons. The options to 
    enter text, change forms, and making sure the user enters valid input will 
    have been implemented as well. We will have the normalized database schema 
    completed, setting the layout for the information stored in the database by
    the application. \\

\textbf{Tuesday, October 21, 2014}
\\	\indent By this time, there will be user authentication and encryption. The 
    information for all the user profiles will also be uploaded into the 
    database. We will also have created several different `likes' and `dislikes'
    that users can choose to add to their profile, which will then be updated 
    in the database.\\

\textbf{Tuesday, November 4, 2014}
\\	\indent At this point, all the information from the users stored in the database 
    can now be queried for `likes' and `dislikes'; a user can choose a `like' 
    or `dislike', and the database will return a list of the current users who
    also have that same interest or disinterest. The user will also be able to
    create groups of different people, called 'circles', that will be saved 
    into a group database. There will also be standard groups created by us 
    that will be gathered from users based on their location, schedule, and 
    other information.\\

\textbf{Tuesday, November 18, 2014}
\\	\indent The user will now be able to create meetings and invite either one of his
    own custom groups or one of the standard groups to the event. There will
    also be the option to input their course schedule. When the user is in
    class, it will display that they are busy on the user's profile.\\

\textbf{Tuesday, December 2, 2014}
\\	\indent Whenever a user is invited to an event, they will get a notification sent 
    to their phone. If the user has entered the schedule and is currently in 
    class, their notification will be delayed until they are free; if they have
    their options set to not receive invitations, no notification will be sent.
    We will also have implemented a map which shows nearby users who have the
    hidden mode disabled in their settings; hidden mode allows a user to accept
    invites and create events without being visible on the map to other users.\\

\newpage
\bibliography{capstone_proposal}
\bibliographystyle{plain}
\nocite{}
\end{document}
