\documentclass[11pt]{article}
\bibliographystyle{plain}
\usepackage{graphicx}
\usepackage[margin=.75in]{geometry}
\usepackage{amsmath}
\usepackage{comment}
\usepackage{setspace}
\usepackage{indentfirst}
\usepackage{hyperref}
\usepackage{url}
\doublespace

\begin{document}
\title{\#FriendFinder}
\author{Tyler Allen, Karen Dana}
\maketitle

% What is it
\section{Overview}


The goal of this capstone is to develop a social networking application for 
the Android mobile platform. The application will allow users to connect with other people and 
plan events. Users will be able to find people by name or through mutual interests. The application
will allow users to invite other users to events they are planning. The application will display 
friends and nearby users. This will provide students with a fun and easy-to-use social networking
application.\\


The application will allow users to register an account with a user name and password. 
Account information will be stored in a database on a back end server. Users will authenticate 
using their user names and passwords every time they launch the application. User account 
authentication will be handled by the server to help ensure secure log in. User account 
credentials will be encrypted before they are sent from the application to the
server. \\


The application will provide an interface for the user to create a 
\textit{user profile}. User profiles will provide a way for users to identify other users 
that they may already know, or that they may wish to become friends with by exposing some of the 
personal information, such as their name and university, to other users. Every user will create a 
user profile when they log into the application for the first time. A user is not required to 
enter information for all input fields in the user profile creation interface. This information will 
be stored in the back-end database. Input fields for user profiles may include
their name, university, course schedules, and an identifying picture of their choosing. The list of 
potential user input fields may be expanded in the future. Users will 
be free to modify/delete information from their user profile at any time. Users will have access
to a search feature that allows them to find other users. The search feature will be implemented using queries
to the database. \\

Users may include their class schedule as part of their profile information. A user's profile will display
a "busy" status if they have a class when the profile is accessed. A user's schedule will only be 
visible to other users if they choose to make their schedule publicly accessible. A calendar 
Application Programming Interface (API) will be used to create the events and store the user's schedule. \\ 

The application will contain three other entities existing alongside users: 
\textit{groups}, \textit{likes}, and \textit{events}. \textit{Groups} are entities that contain zero or more other 
\textit{groups}, and zero or more users. The current plan is to create \textit{default groups} that users may join. 
For example, a user could join the "Western Carolina University" group if they attend Western 
Carolina University. Users will also be able to create custom groups that other users may join.
A user's \textit{group membership} will be visible from their profile as identifying 
information. A \textit{like} is anything in which the user has an interest.
For now, users will be able to choose from a list of predefined
\textit{likes}. \textit{Likes} will show up on a user's profile, and users will be able to search for users who have
a particular \textit{like}. Users are permitted to create \textit{events} within the application. 
They can invite other users and entire groups to attend the \textit{event}. Every \textit{event} will have an \textit{event 
page} containing the name, date, time, length, location of the \textit{event}, as well as a list of attending users. The current 
plan is to have the application display a notification on an attending user's phone when an \textit{event} is about to 
start. Users will be able to see a scrolling list of information about groups and events. 
The search feature will be capable of searching for groups and users who have specific likes. \\

%Quitting here 9/4/2014

The application will allow users to share their locations with their friends and chosen users. Users
will be able to identify nearby users within a predefined radius and treat them as a \textit{group}. 
Scheduled events will contain a map that displays a map marker at the event location. 
The current plan is to use a Global Positioning System (GPS) service to locate users. The application 
will use a mapping API, such as Google Maps, to display user or event locations. \\

The application will be targeted at Android Version 4.0. This version was chosen because the majority 
($85.7\%$) of Android devices are running Android Version 4.0 and above according to Android Developer
statistics generated on August $12^{th}$, 2014 \cite{android_stats}. The current plan is to use an 
Structured Query Language (SQL) database as the backend database and access the database from another high level language 
a database connector API such as Java Database Connectivity (JDBC). The software architecture model for communication between the 
mobile application and the server will be based on the client-server model. A data encryption API
will be used to encrypt user credentials when they are transferred from the application to the 
server as a way to increase user security.

%Why is it a good capstone
\section{Problem Statement}

This project contains a number of challenging aspects. A user authentication system involves sending 
encrypted user credentials from the application to the server, querying the database to check the 
validity of the credentials, and maintaining an authenticated connection between the server and 
the application. An encryption API will need to be learned and applied to authenticate the
user credentials. An Android application interfacing with a server will require large amounts of 
dynamic content to be generated by the application in order to provide current content; this can be 
difficult due to frequent changes and inconsistent behavior in the Android API. The application 
includes features that locate devices within a certain range using GPS; this can be difficult to 
accomplish in a timely and resource efficient manner and will require a large amount of prior research. 
The server will also need to sync with all authenticated users periodically to keep information about
groups and events up to date without causing unnecessary web traffic and draining the battery life 
of devices running the application.


\section{Requirements Specification}

\begin{itemize}
\item The application should provide the ability to create a user profile. 
    We will investigate using facebook authentication 
for this app because it is popular and provides an easy method of user registration \cite{capstone_facebook}.

\item The application will allow a user to join \textit{groups}, add \textit{likes} to 
    their profile, and broadcast/receive events based on these \textit{likes} and \textit{groups}. 
Users will be able to create \textit{likes}/\textit{groups}.

\item The application will allow users to search for other users, \textit{likes}, and \textit{groups}
    available. Users will also be able to build their schedule. User schedules
    will allow the application to deactivate itself when the user is `busy' 
    according to their schedule.

\item Users will be able to have a `friends' list, a special group which allows 
viewing of additional profile information such as the user's class
schedule. This will also allow the user to broadcast their location to their 
friends. GPS is notoriously known for using large amounts 
of power, draining the battery in a short amount of time. We will be investigating
methods for efficiently making use of the GPS transmitter as well as using 
other technologies in place of GPS in order to conserve battery power \cite{conserve_gps} \cite{gps_alternatives}.

\item  Users will have a scrolling list containing event and group information. 
The server should provide a stable, consistent database of user credentials, 
profiles, and other information.

\item The server will be able to provide information about a user without revealing 
information marked private (i.e. extract `busy' status from a user's schedule and 
deliver it to other users, 
without revealing the actual details of the schedule).

\item There will be a user authentication mechanism; the user credentials will 
be encrypted by an encryption API.
\end{itemize}





%TODO: Can't remember - review tomorrow
\section{Testing Plan}

    Testing will be completed as the application is developed. We will create 
    regression/integration tests using Android TestSuite - a derivative of 
    JUnit. An adequate server-side unit testing framework will be selected 
    based on the final language decision; presently, there is no concrete 
    language decision for the server while the concerns about 
    performance, usability, and database and API availability are weighed.  Testing will 
    also include an evaluation of the user experience on Android devices 
    of multiple form factors and screen sizes. Ideally, a large number of registered users 
    can be simulated to test usability. The server will be located on Polaris. 
    Test Android devices will include Samsung Galaxy Neo, an LG G2, 
    a Google Nexus 7 (2012), and any other devices that can be procured during 
    the course of this project. All testing devices will run Android 4.0 or higher. 

%TODO: Can't remember
\section{Schedule of Completion}

\textbf{Tuesday, September 16, 2014}
\\
\indent  The planning stage for the capstone
will be completed; this includes the layouts for the application
and Unified Modelling Language (UML) diagrams for the entire `Friend Finder' project.\\


\textbf{Tuesday, September 30, 2014}
\\  \indent  The skeleton of the application 
    will be complete. The skeleton contains the page layouts, including the 
    graphics, styling, and non event-driven buttons. Options to 
    enter text and edit user profiles and events will have been completed. User input validation will 
    be completed as well. The normalized database schema will be 
    completed, setting the layout for the information stored in the database by
    the application. \\

\textbf{Tuesday, October 21, 2014}
\\	\indent User authentication and encryption will be completed. The 
    database schema will be applied to the database and the database will be set 
    up for queries and modification. There will be several different default \textit{likes} 
    that users can choose to add to their profile, which will then be updated 
    in the database.\\

\textbf{Tuesday, November 4, 2014}
\\	\indent The information gathered from user profiles
    that is stored in the database can now be queried for \textit{likes}; a user can choose a 
    \textit{like}, and the database will return a list of the current users who
    also have that same interest. Users will also be able to
    create groups of different people that will be saved 
     in the back-end database. Default groups will have been completed.  \\

\textbf{Tuesday, November 18, 2014}
\\	\indent Users will now be able to create events and invite their friends or groups to the event. 
    The user profile creation interface will be complete. When the user is in
    class, it will display that they are busy on the user's profile.\\

\textbf{Tuesday, December 2, 2014}
\\	\indent When a user is invited to an event, the application will generate a
    notification for the user. If the user has entered their class schedule and is currently in 
    class, their notifications will be silenced until they are free; if they have
    their options set to not receive notifications, no notification will be created.
    We will also have completed a map feature which shows nearby users who have the
    hidden mode disabled in their settings; hidden mode allows a user to accept
    invites and create events without being visible on the map to other users.\\

\newpage
\bibliography{capstone_proposal}
\nocite{}
\end{document}
